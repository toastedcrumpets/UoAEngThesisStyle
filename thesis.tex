% Everything after a percentile (%) is a comment and is ignored by LaTeX

%%%%% UoAThesis options
% MEng thesis - you don't want to use the doublesided option as it
% "wastes" pages every chapter. 
%
% PhD thesis - you might want to add the doublesided option, it looks
% nicer.
%
\documentclass[12pt]{UoAThesis}
\thesistitle{Style template for Aberdeen Engineering and a rough guide to \LaTeX}
\thesisauthor{Marcus Nigel Campbell Bannerman}
\thesispdfkeywords{}
\thesisyear{2017}

\thesisabstract{This is a tutorial on \LaTeX in the style of a
  thesis. It shows you how to use the UoAThesis style, and some common
  formatting elements of \LaTeX. Please note, your supervisor is the
  ultimate authority on the style and formatting of your thesis!}

\acknowledgements{I would like to dedicate this work to .....}

% The bibtex file where your references are stored
\bibliography{thesis}

\begin{document}

\chapter{Introduction}
Welcome to a latex thesis style for engineering students working on
their honours project. I'll illustrate how to perform certain cool
features of this thesis in the following text.

First thing to note. In order for things such as the table of
contents, references, and {\bf nomenclature} to work, you need to run
latex multiple times, including running biber (bibliography)
and makeindex (nomenclature). The command I use to make the
nomenclature is as follows:

{\tt
  makeindex thesis.nlo -s nomencl.ist -o thesis.nls
}

How you run these depends on whether you're on windows, Mac, or Linux,
so please google ``latex nomencl'' or ``latex biber'' and be sure to
add the platform (windows/mac/linux) and the editor (e.g., miktex) if
you're using one!

The second thing to note is that to start a new paragraph I just
leave at least one blank line in the text!

\section{Numbering of Sections, Tables, and Equations}
The first major benefit of Latex is that everything is numbered
automatically. You don't have to generate a table of contents,
figures, references or even a nomenclature. This is all handled
automatically.

\section{\label{sec:math} Mathematical Expressions}
An excellent feature of latex is its support for math. For example, we
can create an equation like so,
\begin{equation}
  y = mx+c.
\end{equation}
Notice how the equation is nicely formatted and numbered? Also notice
that the quation is punctuated like it is a sentence and part of the
text (because it is!). We can even automatically refer to the equation
number. First, when we make the equation we have to give it a label
like so,
\begin{equation}\label{eq:integralofc}
  a = \int c(x)dx.
\end{equation}
Then, in the text we can refer to it just by typing
Eq.~\ref{eq:integralofc}! We can put labels anywhere, in figures,
tables, chapters wherever, and use the ref command to refer to
them. For instance, this section is Sec.~\ref{sec:math}. Notice the
label just after the section command above?

We can also write math inline. So for example I might say, our first
function is $y=mx+c$. This is the same function as above, but printed
in-line.

There are a million math commands, so please take a look at the AMS
math package manual for more information. Here is an impressive 
equation to show off the features,
\begin{multline}
  \left( \frac{\partial}{\partial t} + \sum_{j}^n \left[ i{\mathcal
        L}^0_j - \sum_{k > j}^n {\mathcal T}(j,k) \right]\right)
  f({\Gamma}_1,\dots,{\Gamma}_n,t)
  \\
  = \sum_{j=1}^{n} \int {\rm d}{\Gamma}_{n+1}\, {\mathcal T}(j,n+1)\,
  f({\Gamma}_1,\dots,{\Gamma}_{n+1},t).
\end{multline}

\section{Adding Figures}
To add figures to your text, you need to use a series of commands, but
you can just copy paste the one below and tweak it for your needs.
\begin{figure}[htp]
  \centering
  \includegraphics[clip,width=0.5\linewidth]{figures/testfig}
  \caption{\label{fig:testfig} A test figure. This caption is wayy too
    short. It should be long enough to explain where the figure came
    from and exactly what it presents. Don't use single sentence captions!}
\end{figure}

I can refer to this figure (Fig.~\ref{fig:testfig}) using the ref
command again, provided I've placed a label in the caption. But notice
how I don't get to choose where it is placed? I can only give hints,
via the [htp] option (which means either place it [H]ere, at the [T]op
of the page, or a whole [P]age). Latex is free to place it where it
thinks best.

The one time we can be forceful about a figure is when it is placed on
its own page using just the [p] placement. But if we're doing a
full-page figure we might want it rotated too to make sure we fully
use the space. See Fig.~\ref{fig:testfig2} for what I mean.
\begin{sidewaysfigure}
  \centering
  \includegraphics[clip,width=0.8\textwidth]{figures/testfig}
  \caption{\label{fig:testfig2} A full page test figure and even the caption is rotated!.}
\end{sidewaysfigure}

\section{Formatting Text}
You can format text in latex, but {\bf please use it sparingly!}. The
results {\em aren't always pleasant}. Use it only when you
\underline{really} want to.

\section{Using a Nomenclature}
To use a nomenclature you must add extra commands throughout your text
whenever you define a new term. For example, a handy acronym is MD
\nomenclature[A MD]{MD}{Molecular Dynamics}. 

The nomenclature command has 3 arguments. The last argument is the
description of the symbol/acronym. The second argument is the
symbol/acronym. The first is the type of entry it is (A=acronyms,
O=operator, V=Variables/Constants, S=Notation, and F is functions),
followed by a space then the alphabetical version of the term. For
example, if I wanted to define a greek symbol such as $\alpha$
\nomenclature[V alpha]{$\alpha$}{Alpha}, I would need to write alpha
here.

\section{Tables}

Tables are made in the following way...
\begin{table}
  \begin{center}
    \begin{tabular}{|p{4cm}|p{2cm}|||p{4cm}|p{2cm}|}
      Surface & $\varepsilon$ & Surface & $\varepsilon$ 
      \\\hline\hline%
      Asbestos, Board & 0.96 & Carbon, Candle soot & 0.95 \\
      Water & 0.95--0.96 & Carbon, Lampblack & 0.95\\
      Iron and steel, rough & 0.94--0.97 & Brick, Red, rough & 0.93\\
      White enamel paint & 0.906 & Iron and steel & 0.74\\
      Iron and steel, Sheet steel, oxidised & 0.657 & Oxidised Lead  & 0.28\\
      Iron and steel, Polished iron & 0.14--0.38 & Rough Aluminium & 0.06\\
      Polished Aluminium & 0.04 & Highly polished Gold & 0.02--0.35
    \end{tabular}
  \end{center}
  %% In square brackets is the short caption name! Always use this if
  %% your caption is longer than one line in the list of tables!
  \caption[Typical values of the emissivity for various materials]{
    Typical values of the emissivity for various materials, 
    taken from Coulson \& Richardson, Vol.~1, Fig.~9.40.}
\end{table}

Just after the begin tabular command, the number of columns and their
alignments are selected. p means the column is a fixed width
paragraph, l c and r are a left aligned, centered and right aligned
columns respectively. The pipe character | adds vertical lines between
columns. Horizontal lines are added using hline commands in the table
itself.

We can also make full-page sideways tables when there is a lot of data
to present (See Table.~\ref{tab:BigTable})
\begin{sidewaystable}
  \begin{center}
    \begin{tabular}{|p{6.5cm}|p{2.0cm}|||p{4.5cm}|p{2.5cm}|}
      Surface & $\varepsilon$ & Surface & $\varepsilon$ 
      \\\hline\hline%
      Asbestos, Board & 0.96 & Carbon, Candle soot & 0.95 \\
      Water & 0.95--0.96 & Carbon, Lampblack & 0.95\\
      Iron and steel, rough & 0.94--0.97 & Brick, Red, rough & 0.93\\
      White enamel paint & 0.906 & Iron and steel & 0.74\\
      Iron and steel, Sheet steel, oxidised & 0.657 & Oxidised Lead  & 0.28\\
      Iron and steel, Polished iron & 0.14--0.38 & Rough Aluminium & 0.06\\
      Polished Aluminium & 0.04 & Highly polished Gold & 0.02--0.35
    \end{tabular}
  \end{center}
  %% In square brackets is the short caption name! Always use this if
  %% your caption is longer than one line in the list of tables!
  \caption[Typical values of the emissivity for various materials (sideways)]{
    \label{tab:BigTable}Typical values of the emissivity for various materials, 
    taken from Coulson \& Richardson, Vol.~1, Fig.~9.40.
}
\end{sidewaystable}

\section{Footnotes}
Footnotes are really easy to do\footnote{And you can even put mathematics (even figures and tables) in them 
  \begin{align}
    f=m\,x+c
  \end{align}
  
  Just don't go crazy and write paragraphs and paragraphs of text in
  them. 
}.

\section{Referencing and Generating}
You can reference entries in your bib file using the key you have set
for it like so~\cite{Bannerman_2009}. I can even do cool things like
say the author of that citation is \citeauthor{Bannerman_2009} and it
was published in \citeyear{Bannerman_2009}. Or even ask for a full
citation, like so: \fullcite{Bannerman_2009}.

\chapter{A Little Further Down The Rabbit Hole}
\lettrine{W}{hy} stop there? There are some lovely features of latex
we can use, like the lettrine at the start of this chapter. Or the
epigraph at the start of the next section!
\section{Epigraphs}
\epigraph{{\em ``Epigraphs are cool!''}}{M.~Campbell Bannerman}

You can use epigraphs to include relevant information, such as quotes
or information on where the section/chapter was published.

\section{Hyperlinks}
Finally, we can even take advantage of the digital nature of our
document and include
hyperlinks. \href{https://marcusbannerman.co.uk/images/stories/pdfs/thesis.pdf}{
  This entire sentence (try clicking it) is linked to my own website}.

\printbibliography[heading=thesisChapterBib]
\end{document}
